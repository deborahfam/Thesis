\chapter{Resultados}\label{chapter:results}
En este capítulo se presenta un marco experimental para evaluar la efectividad de los experimentos realizados. Para ello se emplean gráficos para visualizar la pérdida y la precisión tanto de entrenamiento como de validación a lo largo de las épocas (iteraciones), marcando la época con menor pérdida de validación y mayor precisión de validación.

Se genera también un gráfico de barras que muestra la distribución de errores por clase en el conjunto de pruebas. Además se genera una matriz de confusión y un informe de clasificación que incluye precisión, recuperación (recall), puntuación F1 y soporte para cada clase. Al final del entrenamiento, se evalúa el modelo en el conjunto de pruebas y se obtiene la precisión del mismo. El modelo con mayor eficiencia luego de varios ajustes tuvo una eficacia cercana al 87\%.

La tabla 4.1 al final del capítulo describe las métricas utilizadas para la evaluación del modelo:

\section*{Pérdida y presición (loss y accuracy)}

%annadir introduccion

En el primer experimento se observa que la precisión de entrenamiento aumenta con cada epoch, la pérdida de entrenamiento disminuye consistentemente y la tasa de aprendizaje permanece constante al principio y luego disminuye para afinar el entrenamiento a medida que el modelo comienza a converger, lo cual indica que el modelo esta entrenando de forma correcta. Sin embargo la división asimétrica de datos en este experimento influye en el aprendizaje del modelo. El uso de división asimétrica puede estar causando que el modelo esté sesgado hacia clases con más muestras, afectando la precisión general y la capacidad de generalizar. Esto se evidencia dado que a pesar de la disminución de la pérdida de validación y el aumento de la precisión de validación, la notable diferencia entre la precisión de entrenamiento y la precisión de validación podría indicar un potencial sobre-ajuste.

    \begin{table}[ht]
      \small
      \begin{center}
          \begin{tabular}{|c|c|c|c|c|c|c|c|} \hline
          E & Loss & Acc & V loss & V acc & LR & M & Batch \\ \hline
          1 & 9.587 & 40.581 & 8.95658 & 56.800 & $10^{-2}$ & acc & 85.25 \\ \hline
          2 & 7.798 & 67.615 & 7.67235 & 66.800 & $10^{-2}$ & acc & 21.72 \\ \hline
          3 & 6.884 & 79.340 & 6.96014 & 69.600 & $10^{-2}$ & acc & 22.56 \\ \hline
          4 & 6.214 & 87.365 & 6.35865 & 71.200 & $10^{-2}$ & acc & 25.81 \\ \hline
          5 & 5.646 & 91.690 & 5.94812 & 75.200 & $10^{-2}$ & vloss & 23.08 \\ \hline
          6 & 5.172 & 92.999 & 5.44954 & 76.800 & $10^{-2}$ & vloss & 23.23 \\ \hline
        %   7 & 4.735 & 94.479 & 5.06016 & 76.400 & $10^{-2}$ & vloss & 23.19 \\ \hline
        %   8 & 4.334 & 96.528 & 4.73837 & 76.800 & $10^{-2}$ & vloss & 22.90 \\ \hline
        %   9 & 3.969 & 97.211 & 4.33689 & 77.200 & $10^{-2}$ & val\_loss & 22.74 \\ \hline
        %   10 & 3.631 & 98.008 & 4.15826 & 74.000 & $10^{-2}$ & val\_loss & 22.56 \\ \hline
        %   11 & 3.315 & 98.406 & 3.89153 & 73.600 & $10^{-2}$ & val\_loss & 23.11 \\ \hline
          \dots & \dots & \dots & \dots & \dots & \dots & \dots & \dots \\ \hline
          34 & 0.627 & 99.886 & 1.24470 & 76.800 & 0.00013 & val\_loss & 23.30 \\ \hline
          \end{tabular}
          \caption{Estadísticas básicas del modelo del experimento 1.}
      \end{center}\label{fig:estadisticas_p1}
  \end{table}

  En el experimento 2 se utiliza la estratificación de datos para garantizar que cada clase este representada de manera proporcional. Similar al Experimento 1, en el Experimento 2 también se observa una disminución constante en la pérdida de entrenamiento (Loss) con cada epoch. La pérdida de entrenamiento (Loss) muestra una disminución significativa desde la primera hasta la última epoch registrada, pasando de 8.418 a 0.406. Esto indica un aprendizaje efectivo y una mejora continua en la capacidad del modelo para predecir con precisión las clases. La precisión (Acc), que comienza en 48.371\% y alcanza el 98.057\%, corrobora esta mejora constante. A diferencia del Experimento 1, la pérdida de validación (V loss) en este experimento, aunque también muestra una tendencia descendente, tiene una alineación más estrecha entre la precisión de entrenamiento y la precisión de validación. La pérdida de validación disminuye de manera más consistente y la precisión de validación es comparativamente más alta que en el Experimento 1, lo que indica una mejor capacidad de generalización.
    
\begin{table}[ht]
    \small
    \begin{center}
        \begin{tabular}{|c|c|c|c|c|c|c|c|} \hline
        E & Loss & Acc & V loss & V acc & LR & M & Batch \\ \hline
        1 & 8.418 & 48.371 & 7.41700 & 66.911 & $10^{-2}$ & accuracy & 184.55 \\ \hline
        2 & 6.362 & 69.943 & 5.81767 & 69.907 & $10^{-2}$ & accuracy & 107.94 \\ \hline
        3 & 5.110 & 77.686 & 4.76153 & 72.969 & $10^{-2}$ & accuracy & 106.30 \\ \hline
        4 & 4.181 & 81.371 & 3.86405 & 78.362 & $10^{-2}$ & accuracy & 104.60 \\ \hline
        5 & 3.417 & 84.771 & 3.25588 & 77.097 & $10^{-2}$ & accuracy & 101.90 \\ \hline
        6 & 2.777 & 87.314 & 2.70253 & 78.495 & $10^{-2}$ & accuracy & 101.85 \\ \hline
        % 7 & 2.247 & 90.143 & 2.24392 & 80.360 & $10^{-2}$ & val\_loss & 101.98 \\ \hline
        % 8 & 1.819 & 91.143 & 1.98878 & 77.364 & $10^{-2}$ & val\_loss & 101.74 \\ \hline
        % 9 & 1.464 & 92.543 & 1.70207 & 77.896 & $10^{-2}$ & val\_loss & 101.27 \\ \hline
        % 10 & 1.197 & 94.000 & 1.43860 & 81.225 & $10^{-2}$ & val\_loss & 101.71 \\ \hline
        % 11 & 0.992 & 94.000 & 1.24007 & 81.358 & $10^{-2}$ & val\_loss & 101.23 \\ \hline
        \dots & \dots & \dots & \dots & \dots & \dots & \dots & \dots \\ \hline
        22 & 0.406 & 98.057 & 0.81854 & 83.955 & 0.00006 & val\_loss & 101.18 \\ \hline
        \end{tabular}
        \caption{Estadísticas básicas del modelo del experimento 2.}
    \end{center}\label{fig:estadisticas_p2}
\end{table}
 
Las tablas 4.1 y 4.2 corresponde a la evaluación del experimento 1 y 2 respectivamente.

\section*{Curva de aprendizaje}

Los análisis expuestos también se ven evidenciados en los gráficos de curvas de aprendizaje generados. Cada figura a continuación muestra dos gráficos, uno de pérdida y otro de precisión respectivamente, a lo largo de los epoch de entrenamiento y validación de cada experimento.

En el primer experimento la pérdida de entrenamiento (línea roja) y la pérdida de validación (línea verde) disminuyen con el tiempo, lo que indica que el modelo está aprendiendo. La mejor epoch basada en la pérdida de validación es la 31, marcada por un punto azul. La precisión de entrenamiento (línea roja) es casi perfecta, cercana al 100\%, lo que puede ser un indicador de sobre-ajuste. La precisión de validación (línea verde) mejora pero tiene una variabilidad considerable y alcanza su punto más alto en la epoch 29, también marcada con un punto azul. Hay una brecha notable entre la precisión de entrenamiento y validación, lo que puede ser un signo de que el modelo no está generalizando bien.

\begin{figure}[ht]%
    \begin{center}
        \includegraphics[width=1\textwidth]{./Graphics/training&validation_p1.png}
        \caption{Curva de aprendizaje a lo largo del proceso de entrenamiento del experimento 1.\label{fig:training_validation_loss}}
    \end{center}
\end{figure}

Similar al primer gráfico, en el segundo la pérdida de entrenamiento y validación disminuye, lo que es positivo. La mejor epoch basada en la pérdida de validación es la 19, que ocurre antes que en el primer gráfico, lo que indica una convergencia más rápida. La precisión de entrenamiento también es alta, pero no tan cercana al 100\% como en el primer gráfico, lo que sugiere un menor riesgo de sobre-ajuste. La precisión de validación muestra menos variabilidad y una alineación más cercana con la precisión de entrenamiento, lo que es un indicador de mejor generalización. La diferencia entre la precisión de entrenamiento y validación es menor en comparación con el primer gráfico, lo que sugiere que el modelo podría estar generalizando mejor.

\begin{figure}[ht]%
    \begin{center}
        \includegraphics[width=1\textwidth]{./Graphics/training&validation_p3.png}
        \caption{Curva de aprendizaje a lo largo del proceso de entrenamiento del experimento 2.\label{fig:training_validation_loss_p2}}
    \end{center}
\end{figure}

Las figuras 4.1 y 4.2 corresponde a la curva de aprendizaje del experimento 1 y 2 respectivamente.

%-----------------------------------------------------------------------------------
\section*{Estadísticas de eficacia}\label{sub:accuracy_statistic_p1}
%-----------------------------------------------------------------------------------
La matriz de confusión proporciona información valiosa sobre el rendimiento del modelo en relación de Actual/Predicho, en términos de su capacidad para clasificar correctamente cada una de las siete clases de cáncer de piel. La diagonal principal de la matriz representa los verdaderos positivos (TP), el número de casos en los que el modelo ha predicho correctamente la clase correspondiente. Los valores fuera de la diagonal principal indican errores de clasificación

\begin{figure}[ht]
    \centering
    \begin{minipage}{0.45\textwidth}
        \centering
        \includegraphics[width=\textwidth]{./Graphics/confussionmatrix_p1.png}
        \caption{Estadísticas de eficacia del modelo al estimar los resultados en el conjunto de pruebas}
        \label{fig:confussion_matrix_p1}
    \end{minipage}\hfill
    \begin{minipage}{0.45\textwidth}
        \centering
        \includegraphics[width=\textwidth]{./Graphics/confussionMatrix_p3.png}
        \caption{Estadísticas de eficacia del modelo al estimar los resultados en el conjunto de pruebas}
        \label{fig:confussion_matrix_p3}
    \end{minipage}
\end{figure}

En ambas matrices la clase con el mayor número de verdaderos positivos es "NV", en la primera con 134 casos predichos correctamente y en la segunda con 862 casos.

En el caso de "AKIEC", la tasa de TP mejoró significativamente, pasando de 7 de 300 a 47 de 500. Incluso teniendo en cuenta el aumento del tamaño del conjunto de datos, se trata de una clara mejora. Se observan mejoras similares en otras clases, como "BCC", "BKL", "DF", "MEL" y "NV". La tasa de TP (true positive o verdaderos positivos) ha aumentado no sólo en términos absolutos, sino también proporcionalmente si se tiene en cuenta el mayor tamaño del conjunto de datos. 'VASC' es un caso especial; mientras que la primera matriz no muestra ningún TP y tiene 3 FN (false negative o falsos negativos), la segunda matriz, a pesar de tener un gran número de FP(false positive o falsos positivos) (22), muestra que el modelo ha empezado a reconocer esta clase, cosa que antes no hacía.

En las siguientes figuras se evidencia el margen de error de las clases con mas errores que se obtuvieron en el experimento 1 y 2 respectivamente.

\begin{figure}[ht]%
    \begin{center}
        \includegraphics[width=0.8\textwidth]{./Graphics/errorByClass_p1.png}
        \caption{Gráfico de errores por clase en el conjunto de pruebas\label{fig:class_errors_p1}}
    \end{center}
\end{figure} 

\begin{figure}[ht]%
    \begin{center}
        \includegraphics[width=0.8\textwidth]{./Graphics/errorByClass_p3.png}
            \caption{Gráfico de errores por clase en el conjunto de pruebas\label{fig:class_errors_p3}}
    \end{center}
\end{figure}

En el caso de "NV", el número de errores del primer gráfico de barras se correlaciona con un conjunto de pruebas en el que el modelo tiene casi las mismas posibilidades de hacer una predicción correcta que de dar un falso positivo. En el segundo gráfico, a pesar del aumento de errores, el modelo tiene la misma proporción de verdaderos positivos que de falsos positivos, lo que sugiere que el rendimiento del modelo se ha mantenido constante en relación con el tamaño del conjunto de datos. MEL" y "BKL" muestran un aumento de los errores en el segundo gráfico de barras, pero también es proporcional al aumento del tamaño del conjunto de datos. La proporción de verdaderos positivos frente a falsos positivos sigue siendo la misma.
AKIEC" presenta un número relativamente pequeño de errores en el segundo gráfico de barras, lo que puede deberse al rendimiento relativamente bueno del modelo en esta clase en el conjunto de datos más grande.

\section*{Informe de clasificación}

La tabla siguientes proporcionan una visión cuantitativa de la precisión, el recall (sensibilidad), el puntaje F1 y el soporte (número de muestras verdaderas) (NM) para cada categoría diagnóstica evaluada. Estos indicadores de rendimiento son esenciales para comprender la capacidad del modelo para identificar correctamente cada condición, así como su confiabilidad general en un conjunto de datos diverso. La métrica de 'Accuracy' refleja la proporción general de predicciones correctas, mientras que los promedios 'Macro' y 'Weighted' proporcionan una perspectiva agregada del rendimiento del modelo, teniendo en cuenta el desequilibrio en el soporte de las clases. 

\begin{table}[ht]
    \small
    \centering
    \caption{Informe de clasificación combinado para los Experimentos 1 y 2}
    \label{tab:classification_report_combined}
    \begin{tabular}{lcccccccc}
    \hline
    & \multicolumn{4}{c}{\textbf{Experimento 1}} & \multicolumn{4}{c}{\textbf{Experimento 2}} \\
    \cline{2-9}
    \textbf{Categoría} & \textbf{Acc} & \textbf{Recall} & \textbf{F1-Score} & \textbf{NM} & \textbf{Acc} & \textbf{Recall} & \textbf{F1-Score} & \textbf{NM} \\
    \hline
    AKIEC & 0.47 & 1.00 & 0.64 & 7   & 0.82 & 0.96 & 0.89 & 49 \\
    BCC   & 0.59 & 1.00 & 0.74 & 10  & 0.81 & 1.00 & 0.90 & 77 \\
    BKL   & 0.66 & 0.63 & 0.64 & 30  & 0.72 & 0.83 & 0.77 & 165 \\
    DF    & 0.33 & 1.00 & 0.50 & 3   & 0.71 & 1.00 & 0.83 & 17 \\
    MEL   & 0.70 & 0.80 & 0.75 & 35  & 0.63 & 0.85 & 0.73 & 167 \\
    NV    & 0.99 & 0.82 & 0.90 & 163 & 0.98 & 0.86 & 0.91 & 1006 \\
    VASC  & 0.60 & 1.00 & 0.75 & 3   & 0.71 & 1.00 & 0.83 & 22 \\
    \hline
    \textbf{Accuracy} & & & 0.81 & 251 & & & 0.87 & 1503 \\
    \textbf{Macro Avg} & 0.62 & 0.89 & 0.70 & 251 & 0.77 & 0.93 & 0.84 & 1503 \\
    \textbf{Weighted Avg} & 0.86 & 0.81 & 0.83 & 251 & 0.89 & 0.87 & 0.87 & 1503 \\
    \hline
    \end{tabular}
\end{table}

Se añadieron a la tabla 4.3 los campos \textit{Accuracy}, \textit{MacroAvg} y \textit{WeightedAvg} para un mejor análisis de los resultados. 

El campo \textit{accuracy} muestra que la precisión mejora de 0.81 a 0.87, reflejando un aumento en la capacidad general del modelo para hacer predicciones correctas. El Macro Avg (Promedio Macro) aumento de 0.62 a 0.77 en precisión y de 0.70 a 0.84 en la puntuación F1, indicando una mejora en el rendimiento medio del modelo a través de todas las categorías. El Weighted Avg (Promedio Ponderado) muestra un incremento de 0.86 a 0.89 en precisión y de 0.83 a 0.87 en la puntuación F1, mostrando que, teniendo en cuenta el número de muestras (soporte), el rendimiento general del modelo ha mejorado.

\section*{Observaciones generales entre los experimentos}

En general los experimentos demuestran resultados interesantes en la clasificación. El Experimento 2 muestra mejoras generalizadas en precisión, recall y puntuaciones F1 para la mayoría de las categorías. El soporte aumenta significativamente, lo que indica que se evaluó al modelo con más muestras, proporcionando una base más robusta para la evaluación del rendimiento. A pesar de que el soporte es mayor, lo que generalmente hace más desafiante mantener altas métricas, el Experimento 2 demuestra mejoras en las métricas en general. Este, a diferencia del 1, maneja mejor el desequilibrio de clases, lo que se refleja en una mejor diferenciación entre ciertas categorías diagnósticas, lo que es indicativo de un modelo más preciso y confiable.

% \begin{table}[ht]
%     \centering
%     \caption{Informe de clasificación para el Experimento 1}
%     \label{tab:classification_report_exp1}
%     \begin{tabular}{lcccc}
%     \hline
%     \textbf{Categoría Diagnóstica} & \textbf{Precisión} & \textbf{Recall} & \textbf{F1-Score} & \textbf{Soporte} \\
%     \hline
%     AKIEC & 0.47 & 1.00 & 0.64 & 7 \\
%     BCC   & 0.59 & 1.00 & 0.74 & 10 \\
%     BKL   & 0.66 & 0.63 & 0.64 & 30 \\
%     DF    & 0.33 & 1.00 & 0.50 & 3 \\
%     MEL   & 0.70 & 0.80 & 0.75 & 35 \\
%     NV    & 0.99 & 0.82 & 0.90 & 163 \\
%     VASC  & 0.60 & 1.00 & 0.75 & 3 \\
%     \hline
%     \textbf{Accuracy} & & & 0.81 & 251 \\
%     \textbf{Macro Avg} & 0.62 & 0.89 & 0.70 & 251 \\
%     \textbf{Weighted Avg} & 0.86 & 0.81 & 0.83 & 251 \\
%     \hline
%     \end{tabular}
%     \end{table}

% La precisión global del modelo fue del 81\%, con una media ponderada de precisión del 86\%, recall del 81\%, y una puntuación F1 del 83\%. Describir la precisión, el recall, y la puntuación F1 para cada categoría diagnóstica, destacando que 'NV' tuvo la precisión más alta (0.99) y 'AKIEC' la más baja (0.47). El recall fue perfecto (1.00) para 'AKIEC', 'BCC', 'DF', y 'VASC', lo que indica que todas las muestras verdaderas fueron identificadas correctamente, pero la precisión varió, sugiriendo un posible desequilibrio en la identificación de falsos positivos.
    
%     \begin{table}[ht]
%         \centering
%         \caption{Informe de clasificación para el Experimento 2}
%         \label{tab:classification_report_exp2}
%         \begin{tabular}{lcccc}
%         \hline
%         \textbf{Categoría Diagnóstica} & \textbf{Precisión} & \textbf{Recall} & \textbf{F1-Score} & \textbf{Soporte} \\
%         \hline
%         AKIEC & 0.82 & 0.96 & 0.89 & 49 \\
%         BCC   & 0.81 & 1.00 & 0.90 & 77 \\
%         BKL   & 0.72 & 0.83 & 0.77 & 165 \\
%         DF    & 0.71 & 1.00 & 0.83 & 17 \\
%         MEL   & 0.63 & 0.85 & 0.73 & 167 \\
%         NV    & 0.98 & 0.86 & 0.91 & 1006 \\
%         VASC  & 0.71 & 1.00 & 0.83 & 22 \\
%         \hline
%         \textbf{Accuracy} & & & 0.87 & 1503 \\
%         \textbf{Macro Avg} & 0.77 & 0.93 & 0.84 & 1503 \\
%         \textbf{Weighted Avg} & 0.89 & 0.87 & 0.87 & 1503 \\
%         \hline
%         \end{tabular}
%         \end{table}
        
% Resultados Generales: La precisión global del modelo fue del 87\%, con una media ponderada de precisión del 89\%, recall del 87\%, y una puntuación F1 del 87\%.
% Resultados por Categoría Diagnóstica: Enumerar la precisión, el recall, y la puntuación F1 para cada categoría diagnóstica. Resalta que todas las categorías mejoraron en comparación con el Experimento 1, con 'NV' mostrando nuevamente la precisión más alta (0.98) y 'MEL' la más baja (0.63).

\section{Consideraciones Finales}\label{subsubsec:final_considerations}
%-----------------------------------------------------------------------------------

Los resultados obtenidos son prometedores y sugieren que los modelos de aprendizaje profundo tienen un potencial considerable para mejorar la precisión y la eficiencia del diagnóstico del cáncer de piel. Tomando como referencia los resultados del experimento 2, se obtuvo una eficacia de clasificación de 87\%, lo cual es bajo con respecto a métodos más robustos mencionados en el estado del arte pero prometedor teniendo en cuenta que los algoritmos desarrollados utilizando EfficientNet obtienen resultados entre 84\% y 86.5\% de eficacia \brackcite{ali2022multiclass}.

Es notable además la importancia para el dataset específico utilizado (HAM10000) de un balance proporcional en el conjunto de datos, dado que se evidencia en los resultados que la desproporción entre los mismos lleva a errores de sobre-ajuste y a un peor rendimiento del modelo.

\section*{Métricas}

\begin{table}[ht]
    \centering
    \small
    \begin{tabular}{|c|p{10cm}|}
    \hline
    \textbf{Término} & \textbf{Descripción} \\
    \hline
    Epoch & Es una iteración completa sobre todo el conjunto de datos de entrenamiento. \\
    \hline
    Loss (Pérdida) & Es una medida de cuán bien el modelo está realizando sus predicciones. Los valores decrecientes indican una mejora en el aprendizaje. \\
    \hline
    Accuracy (Precisión) & Muestra el porcentaje de etiquetas que el modelo predice correctamente para el conjunto de entrenamiento. \\
    \hline
    V loss (Pérdida de Validación) & Es similar a la pérdida, pero se calcula sobre un conjunto de datos que no se utiliza para el entrenamiento. \\
    \hline
    V acc (Precisión de Validación) & Muestra el porcentaje de etiquetas que el modelo predice correctamente para el conjunto de datos de validación. \\
    \hline
    LR (Learning Rate) & La tasa de aprendizaje dicta cuánto se ajustan los pesos del modelo en cada actualización. \\
    \hline
    Next LR (Próxima Learning Rate) & Indica la próxima tasa de aprendizaje planificada. La adaptación de la tasa de aprendizaje puede ayudar a evitar el estancamiento y mejorar la convergencia. \\
    \hline
    Monitor & Muestra la métrica que se está utilizando para monitorizar el rendimiento del modelo. Cambia de \textit{accuracy} a \textit{val loss}, lo que probablemente indica que el cambio se hizo para evitar el sobre'ajuste. \\
    \hline
    Duration (Duración) & Tiempo que tardó cada epoch en completarse. Importante para evaluar la eficiencia del entrenamiento. \\
    \hline
    \end{tabular}
    \caption{Descripción de términos clave en el entrenamiento de modelos de aprendizaje automático.}
    \label{table:terminology}
    \end{table}