\chapter{Detalles de implementación y experimentos}\label{chapter:implementation}

Este capítulo presenta los métodos, técnicas, detalles e hiperparámetros utilizados en la implementación general del modelo y especifica de cada experimento.

\section{Herramientas y Tecnologías}

La implementación del modelo se llevó a cabo utilizando las siguientes herramientas:

\begin{itemize}
    \item \textbf{Lenguajes de Programación:} Python: utilizado por su rica biblioteca de paquetes de aprendizaje automático, incluyendo TensorFlow y Keras para la construcción y entrenamiento del modelo.
    \item \textbf{Frameworks de Aprendizaje Profundo:} TensorFlow y Keras proporcionan las funcionalidades necesarias para diseñar, entrenar y validar modelos de aprendizaje profundo con alta eficiencia y flexibilidad.
    \item \textbf{Técnicas de Preprocesamiento:} Herramientas para la normalización de imágenes, el aumento de datos, y el balanceo de clases serán utilizadas para preparar el dataset para el entrenamiento del modelo.
    \item \textbf{Optimización y Regularización:} Se integrarán técnicas como el ajuste dinámico del learning rate y la regularización L1 y L2 para optimizar el rendimiento del modelo y prevenir el sobre-ajuste.
    \item \textbf{Hardware y Recursos Computacionales:} Se utilizó Google Collab para acelerar el proceso de entrenamiento del modelo, permitiendo la experimentación con diferentes híper-parámetros y arquitecturas de forma eficiente.
\end{itemize}

\section*{Implementación del modelo}
\subsection*{Hiperparámetros y configuración}

\begin{enumerate}
   \item Modelo Base: EfficientNetB1 con pesos de ImageNet.
   \item Tasa de Aprendizaje Inicial: $0.001$.
   \item Regularizadores: L2 con $\lambda=0.016$ y L1 con $\lambda = 0.006$.
   \item Tasa de Dropout: 45\%.
   \item Epochs: $40$
   \item Callback: Incluye el LRA con parámetros específicos de paciencia, umbral, factor de reducción, y control de \textit{dwell}.
\end{enumerate}

\subsection{Aumento y división de datos}

Para la carga y procesamiento de imágenes se utilizó la clase \textit{ImageDataGenerator} de Keras \brackcite{img_gen}. Esta clase es una parte integral de la biblioteca Keras y proporciona una forma eficiente de manipular imágenes para tareas de aprendizaje automático. Su principal función es facilitar la creación de lotes de imágenes que se utilizan durante el entrenamiento y la evaluación de modelos de Machine Learning, especialmente en el contexto de redes neuronales. 

Las transformaciones aplicadas fueron las siguientes

\begin{table}[ht]
   \centering
   \begin{tabular}{lccc}
   \hline
   Parámetro & Descripción  & Valor \\ \hline
   rotation range & 	Rango de rotación & 20 grados \\
   width shift range & Rango de desplazamiento horizontal & 20\% \\
   height shift range & Rango de desplazamiento vertical & 20\% \\
   shear range & 	Rango de corte & 20\% \\
   zoom range & Rango de zoom & 20\% \\
   horizontal flip & Activación de volteo horizontal & verdadero \\
   fill mode & Modo de relleno para manejar los píxeles faltantes & nearest \\ \hline
   \end{tabular}
   \caption{Parámetros de aumento de datos}
   \label{tab:data_augmentation_params}
   \end{table}
     
\section{Capas adicionales y regularización}
   
   Para ajustar el modelo de EfficientNetB1 a nuestras necesidades, se añaden capas adicionales:
   
   \subsection{Normalización por lotes}
   
   Esta capa se define con un \textit{eje de normalización} establecido en $-1$, lo que indica que la normalización se aplica a lo largo del último eje en el tensor de entrada. Además, se configura un \textit{momentum} de $0.99$ y un valor de \textit{epsilon} de $0.001$. El alto valor de momentum ayuda a mantener la estabilidad de las medias y varianzas móviles a lo largo del entrenamiento, mientras que el pequeño valor de epsilon evita divisiones por cero, asegurando así cálculos numéricos estables \brackcite{regularization}.
   
   \subsection{Capa densa}
   
   Se integra una capa densa con $256$ neuronas, que juega un papel clave en la síntesis de las características aprendidas por el modelo. Se utiliza un regularizador \textit{L2} con un \textit{lambda} de $0.016$ para los pesos de la capa, lo que ayuda a penalizar y controlar el tamaño de los pesos, reduciendo así el riesgo de sobre-ajuste. Además, tanto el regularizador de actividad como el regularizador de bias se configuran con un regularizador \textit{L1} con un \textit{lambda} de $0.006$. Este enfoque impone una penalización en los pesos y los sesgos, promoviendo un modelo más simple y disperso. La función de activación utilizada es \textit{ReLU} , conocida por su eficacia en la introducción de no linealidad en el modelo, lo que permite aprender relaciones complejas entre las características \brackcite{dense}.
   
   \subsection{Dropout}
   
   En la arquitectura del modelo, se integra una capa de dropout para aumentar la robustez y prevenir el sobre-ajuste. Esta capa se configura con una tasa de desactivación del 45\% , lo que significa que, durante el entrenamiento, el 45\% de las neuronas se desactivarán aleatoriamente en cada paso. 
   
   En una primera iteración del algoritmo tuvo una tasa de desactivación más baja. Luego de varias iteraciones se concluyó que se necesitaba un algoritmo de clasificación que estudiara más a detalle la data fomentando así una mejor generalización y reduciendo el riesgo de sobre-ajuste en el proceso de aprendizaje. Basándonos en esto aumentamos la tasa y obtuvimos mejores resultados.
   
   Para asegurar la reproducibilidad, se establece una semilla(seed) $123$. Esta introducción de aleatoriedad ayuda a que el modelo no dependa excesivamente de ninguna característica o neurona específica \brackcite{dropout}.
   
   \subsection{Capa de salida}
   
   La capa de salida utiliza una activación \textit{softmax} para transformar las salidas del modelo en probabilidades de pertenencia a cada clase. Esto, clasifica las entradas en categorías distintas, proporcionando probabilidades para cada clase, lo cual es esencial en la clasificación multi-clase.
   
   \subsection{Optimización}
   
   Para el proceso de entrenamiento, se utiliza el optimizador \textit{Adamax} \brackcite{adamax}. Este es una variante del conocido optimizador Adam, que combina las ventajas de los métodos adaptativos de tasa de aprendizaje con una implementación más robusta en entornos con gradientes dispersos, lo cual es común en imágenes médicas 
   
   La función de pérdida elegida es la \textit{categorical crossentropy} \brackcite{vitalflux_categorical_crossentropy}, idónea para problemas de clasificación multi-clase. Se configura el modelo para minimizarla y se rastrea la precisión como métrica principal.

\section{Distribución de datos}

\subsection{Experimento 1: Evaluación de la eficiencia de la división asimétrica de datos en la clasificación de imágenes de cáncer de piel}

   La distribución de datos fue la siguiente:

   \begin{enumerate}
      \item El 95\% de los datos se destinan al conjunto de entrenamiento.
      \item El 2.5\% de los datos restantes se destinan al conjunto de validación.
      \item El 2.5\% restante se destina al conjunto de pruebas.
   \end{enumerate}
   
   Además se mezclan aleatoriamente los datos y se utiliza una variable fija para garantizar que la división sea reproducible. Aquí se utiliza \textit{dummy split} para mantener la proporción deseada entre validación y prueba. Por lo que este experimento tiene $9514$ datos de entrenamiento, $251$ de test y $250$ de validación.

   \begin{table}[ht]
      \centering
      \begin{tabular}{lccc}
      \hline
      \textbf{Diagnostic category} & \textbf{Training} & \textbf{Validation} & \textbf{Testing} \\
      \hline
      NV       & 300 & 158 & 163 \\
      MEL      & 300 & 25  & 35  \\
      BKL      & 300 & 34  & 30  \\
      DF       & 115 & 5   & 3   \\
      AKIEC    & 300 & 11  & 7   \\
      BCC      & 300 & 16  & 10  \\
      VASC     & 142 & 1   & 3   \\ \hline
      \end{tabular}
      \caption{Experimento 1: Distribución de imágenes de cáncer de piel en los conjuntos de entrenamiento, test y validación}
      \label{table:train_test_validate_e1}
      \end{table}
   
\subsubsection*{Generadores de datos y preprocesamiento}

Se establece para este un tamaño objetivo de muestras por clase ($300$ en este caso), y se utiliza un bucle para iterar a través de cada clase única. Se realiza un re-muestreo con reemplazo para clases con un número de muestras menor al objetivo ($300$), y sin reemplazo para clases con un número igual o mayor al tamaño objetivo.

\begin{table}[ht]
   \centering
   \begin{tabular}{lccc}
   \hline
   Diagnostic category & Sampling  \\ \hline
   AKIEC & 300 \\
   BCC & 300 \\
   BKL & 300 \\
   DF & 115 \\
   MEL & 300 \\
   NV & 300 \\
   VASC & 142 \\ \hline
   \end{tabular}
   \caption{Distribución de muestras por categoría después del sobre-muestreo}
   \label{tab:sampling_distribution_1}
   \end{table}


Como se evidencia anteriormente, al separar la data en clases el conjunto se mantiene desbalanceado. Se hace necesario aplicar un método llamado \textit{Class Weighting} para compensar este desbalanceo. Este método consiste en asignar un peso a cada clase inversamente proporcional a su frecuencia. De esta manera, las clases con menor representación tendrán un mayor peso y las clases con mayor representación tendrán un menor peso. Esto permite que el modelo se entrene de manera más equilibrada y que no se sesgue hacia las clases con mayor representación.

\begin{table}[ht]
   \centering
   \begin{tabular}{lccc}
   \hline
   Diagnostic category & Sampling  & Weighting\\ \hline
   AKIEC & 300 & 1.00\\
   BCC & 300 & 1.00\\
   BKL & 300 & 1.00\\
   DF & 115 & 2.60\\
   MEL & 300 & 1.00\\
   NV & 300 & 1.00\\
   VASC & 142 & 2.11\\ \hline
   \end{tabular}
   \caption{Distribución de muestras con peso asignado}
   \label{tab:weighting_distribution}
   \end{table}


\subsection{Experimento 2: Análisis de la estratificación de datos en la clasificación de imágenes de cáncer de piel}

Descripción: Este experimento explora la estratificación de datos para mantener una distribución uniforme de etiquetas en cada conjunto de datos. La proporción de los conjuntos de datos es más equilibrada en comparación con el Experimento 1, lo que ofrece insights sobre la importancia de la distribución equitativa de datos en el entrenamiento y evaluación de modelos.

La distribución de datos fue la siguiente:

   \begin{enumerate}
      \item El 70\% de los datos se destinan al conjunto de entrenamiento.
      \item El 15\% de los datos restantes se destinan al conjunto de validación.
      \item El 15\% restante se destina al conjunto de pruebas.
   \end{enumerate}
   
Se utiliza \textit{stratify} en ambas divisiones para mantener la distribución de etiquetas \textit{label} en cada conjunto. Se calcula la proporción del conjunto de prueba sobre la suma del conjunto de test y el de validación.

   \begin{table}[ht]
      \centering
      \begin{tabular}{lccc}
      \hline
      \textbf{Diagnostic category} & \textbf{Training} & \textbf{Validation} & \textbf{Testing} \\
      \hline
      NV    & 500 & 1006 & 1006 \\
      MEL   & 500 & 167  & 167  \\
      BKL   & 500 & 165  & 165  \\
      DF    & 500 & 17   & 17   \\
      AKIEC & 500 & 49   & 49   \\
      BCC   & 500 & 77   & 77   \\
      VASC  & 500 & 21   & 22   \\
      \hline
      \end{tabular}
      \caption{Experimento 2: Distribución de imágenes de cáncer de piel en los conjuntos de entrenamiento, test y validación}
      \label{tab:train_test_validate_e2}
      \end{table}
      

Luego este quedaría distribuido en $7010$ datos de entrenamiento, $1503$ de test y $1502$ de validación.

\subsubsection*{Generadores de datos y preprocesamiento}

Se establece también un tamaño objetivo de muestras por clase ($500$). Se utiliza la función \textit{groupby} para agrupar el dataFrame por la etiqueta de clase. Se itera sobre cada grupo, y se realiza un re-muestreo con reemplazo para grupos menores al tamaño deseado y sin reemplazo para los grupos que ya alcanzan o superan el tamaño deseado.

En este, a diferencia del primero, en cada clase se alcanza la misma cantidad de muestras, por lo que no es necesario aplicar \textit{Class Weighting}.

\begin{table}[ht]
   \centering
   \begin{tabular}{lccc}
   \hline
   Diagnostic category & Sampling  \\ \hline
   AKIEC & 500 \\
   BCC & 500 \\
   BKL & 500 \\
   DF & 500 \\
   MEL & 500 \\
   NV & 500 \\
   VASC & 500 \\ \hline
   \end{tabular}
   \caption{Distribución de muestras por categoría después del sobre-muestreo}
   \label{tab:sampling_distribution}
   \end{table}