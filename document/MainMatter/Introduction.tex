\chapter*{Introducción}\label{chapter:introduction}
\addcontentsline{toc}{chapter}{Introducción}

Desde su surgimiento en 2006, el aprendizaje profundo se ha establecido como una importante subdisciplina dentro del aprendizaje automático, especialmente en áreas relacionadas con la percepción visual humana. Esta metodología procesa datos a través de múltiples capas que incluyen estructuras complejas y transformaciones no lineales \brackcite{lecun2015deep}. Actualmente ha logrado avances significativos en áreas como la visión artificial, reconocimiento de voz, procesamiento del lenguaje natural, reconocimiento de audio y bioinformática \brackcite{deng2014deep}. Desde 2013, el aprendizaje profundo se ha reconocido como uno de los diez avances tecnológicos más significativos, dadas sus amplias aplicaciones potenciales en el análisis de datos \brackcite{cai2020review}.

El enfoque del aprendizaje profundo imita la red neuronal del cerebro humano. Utiliza múltiples capas de procesamiento no lineal para abstraer los datos en distintos niveles, obteniendo características abstractas que pueden emplearse para tareas como detección de objetos, clasificación o segmentación. Una de las mayores ventajas del aprendizaje profundo es su capacidad para reemplazar la extracción manual de características con algoritmos eficientes de aprendizaje de características, ya sea de manera no supervisada o semisupervisada, y la extracción jerárquica de características \brackcite{song2013hierarchical}.

El ámbito de la atención médica está íntimamente ligado a la salud humana. Hoy en día, nos enfrentamos a un volumen abrumador de datos médicos, y la capacidad de utilizar eficientemente estos datos es vital para el avance de la medicina. A pesar de la abundancia de información médica disponible, existen numerosos desafíos: la diversidad de estos datos, que abarcan desde imágenes hasta textos, videos y gráficos; la variabilidad en la calidad debido a los diferentes equipos utilizados; la naturaleza dinámica de los datos que cambian con el tiempo y eventos específicos; y las diferencias individuales que hacen que las reglas generales de enfermedades no sean universalmente aplicables \brackcite{zhang2017基于深度学习的医学影像诊断综述}. Además, hay varios aspectos que complican aún más estos desafíos. Dentro de este vasto espectro de datos, las imágenes médicas constituyen un componente esencial \brackcite{cai2020review}.

La dermatología es una rama especializada de la medicina que se centra en el estudio, diagnóstico, tratamiento, y manejo de las enfermedades y trastornos de la piel, el órgano más extenso del cuerpo humano \brackcite{fundacionpielsana_dermatologia}. El cáncer de piel es el tipo de cáncer más frecuente en todo el mundo, y su detección precoz es crucial \brackcite{american_cancer_society_estadisticas_2023}.

\section*{Problemática}

La problemática central en la detección del cáncer de piel radica en la necesidad de mejorar la precisión y rapidez del diagnóstico. Tradicionalmente, esta tarea recae en dermatólogos y el diagnóstico de melanoma, que depende de la evaluación clínica y los hallazgos clásicos en la biopsia de la lesión. La inspección visual puede no ser suficiente para diferenciar lesiones benignas de tumores malignos, y aunque la biopsia de piel es el estándar de oro, es un procedimiento invasivo con limitaciones. Además, la experiencia, el costo y la disponibilidad son desafíos para el uso generalizado de herramientas no invasivas en el diagnóstico clínico \brackcite{das2021machine}. Para esto los algoritmos de aprendizaje profundo constituyen una herramienta de solución valiosa.

\section*{Antecedentes}

Las Redes Neuronales Convolucionales o \textit{Convolutionals Neural Network (CNN)} han sido fundamentales en la identificación de características en imágenes médicas. Su uso en medicina aprovecha estructuras específicas de datos para capturar patrones con mayor precisión \brackcite{unal2023doc}. Es por eso que una rama arduamente investigada en el mundo de la ciencia y la medicina. En 2018, Brinker et al. \brackcite{brinker2018skin} analiza 13 artículos sobre la aplicación de CNN en la clasificación de lesiones cutáneas, destacando su alto rendimiento y la reutilización de CNNs pre-entrenadas. En el aprendizaje profundo orientado al análisis de imágenes dermatoscópicas Ameri et al. (2020) \brackcite{ameri2020deep} marcan un avance significativo al implementar una CNN profunda para procesar imágenes dermatoscópicas directamente, mejorando la eficacia del proceso de clasificación sin necesidad de segmentación previa. Más tarde en 2022, Shetty et al. \brackcite{shetty2022skin} logran una precisión del 95,18\% en la clasificación de lesiones cutáneas utilizando CNN, demostrando la superioridad de estas redes sobre otros algoritmos de ML.

En otras investigaciones enfocadas al diagnóstico mejorado de lesiones pigmentadas, Tajerian et al. (2023) \brackcite{tajerian2023design} presentan un enfoque metodológico para el diagnóstico de lesiones cutáneas pigmentadas utilizando CNN, logrando una alta puntuación F1 de 0,93. En 2021 Adegun et al. \brackcite{adegun2021deep} recoge un conjunto de estudios enfocados en el desarrollo de algoritmos con CNN para la detección de cáncer de piel. Estos incluyen enfoques de segmentación y clasificación, arquitecturas de auto-encoder-decoder, y la implementación de redes pre-entrenadas como AlexNet y VGG16.

En un enfoque distinto pero no menos importante Dildar et al. (2021) \brackcite{dildar2021skin} realizan una revisión exhaustiva para evaluar el impacto de las técnicas de aprendizaje profundo en la detección precoz de cáncer de piel, analizando diversos enfoques y resultados.

Las universidades han desempeñado un papel crucial en la investigación y el desarrollo de tecnologías avanzadas en el campo de la medicina, especialmente en la detección y tratamiento del cáncer. Estos centros académicos no solo proporcionan una base sólida para la investigación teórica, sino que también fomentan la innovación práctica mediante el uso de tecnologías emergentes como el aprendizaje automático y la inteligencia artificial. En particular, nuestra universidad ha contribuido significativamente a este campo.

La tesis de Darien Viera Barredo titulada \textit{Autómata celular estocástico en redes complejas para el estudio de la invasión, migración y metástasis del cancer} \brackcite{automata_celular_thesis} proporciona un marco detallado sobre cómo se aborda el estudio del cáncer desde una perspectiva matemática y computacional avanzada. El modelo propuesto utiliza autómatas celulares estocásticos para simular el crecimiento avascular y vascular del tumor. En el se aborda la complejidad del ciclo vital tumoral, destacando la importancia de su comprensión tanto para la investigación del cáncer como para la salud pública. Tradicionalmente, la modelación matemática y computacional se ha centrado en las etapas tempranas del desarrollo tumoral, donde la mortalidad es baja. Sin embargo, este estudio se enfoca en las fases avanzadas, que son críticas para la vida del paciente.

\section*{Motivación}

La motivación detrás del uso de algoritmos de \textit{machine learning} (ML) para el diagnóstico del cáncer de piel es significativa y valiosa. A diferencia de otros tipos de cáncer, el cáncer de piel se forma en la superficie de la piel y suele ser visible. Esto plantea una oportunidad única para la detección temprana y el tratamiento, lo cual es esencial ya que la mayoría de los casos de cáncer de piel son tratables si se detectan a tiempo \brackcite{american_cancer_society_estadisticas_2023}.

Los algoritmos de ML ofrecen una solución prometedora para analizar imágenes de la piel. Estos algoritmos pueden identificar patrones complejos con una precisión y consistencia mayor que los métodos de diagnóstico humano, reduciendo así la posibilidad de diagnósticos incorrectos debido a la interpretación subjetiva y variable de los expertos \brackcite{das2021machine}. Además, el ML puede procesar grandes volúmenes de datos rápidamente y su capacidad para aprender y adaptarse con el tiempo significa que la detección y clasificación del cáncer de piel puede mejorar continuamente.l \brackcite{das2021machine}.

Si bien ya existen algoritmos eficaces para la clasificación del melanoma, que es un tipo de cáncer de piel, la propuesta de un modelo que pueda clasificar varios tipos de cáncer de piel y generalizar entre ellos es fundamental. Esto permitiría procesar una gama más amplia de imágenes dermatológicas, mejorando potencialmente la precisión en la detección y el tratamiento del cáncer de piel.

\section*{Objetivos}

Este trabajo propone como objetivo fundamental el diseño y validación de un modelo predictivo basado en \textit{deep learning} para el diagnóstico del cáncer de piel mediante la clasificación de imágenes dermatológicas. El modelo que se diseña y se desarrolla clasifica imágenes donde se prioriza la precisión de los resultados y el rango de generalización del modelo. Esto se concibe así con la idea de desarrollar trabajos posteriores que admitan otros datos de entrada.

Entre los objetivos específicos del proyecto se encuentran:

\begin{enumerate}
    \item Estudiar el estado del arte sobre las técnicas empleadas en el diagnóstico de imágenes dermatológicas y su efectividad.
    \item Estudiar modelos ya implementados actualmente y rear un modelo de \textit{deep learning} que dado un conjunto de imágenes de cáncer de piel clasifique el tipo al que pertenecen.
    \item Decidir mediante un algoritmo de machine learning la mejor distribución de datos para entrenamiento del modelo e Implementar mejoras potenciales al modelo e hiperparámetros para mejor precisión del mismo.
    \item Implementar métricas de precisión para evaluar la propuesta con otras del estado del arte.
\end{enumerate}

\section*{Contribuciones}

La metodología propuesta podría contribuir al desarrollo de un sistema de clasificación de imágenes de cáncer de piel más preciso, para luego ser utilizado en la práctica clínica en el diagnóstico de cáncer de piel. De esta forma podrían extenderse los algoritmos de clasificación existentes para el análisis de estos datos. 

\section*{Estructura de la tesis}

El contenido de la tesis se organiza de la siguiente forma. En el capítulo 1 se exponen las principales alternativas presentes en la literatura que se han desarrollado para la clasificación de imágenes. En el capítulo 2 presenta el modelo propuesto para la implementación de un sistema de clasificación de imágenes de cáncer de piel. En los capítulos 3 y 4 se describe  los algoritmos y técnicas utilizadas, se describen los experimentos realizados y se exponen los resultados obtenidos y se analiza la efectividad de estos. Finalmente, se presentan las conclusiones de la tesis y las recomendaciones para investigaciones futuras.