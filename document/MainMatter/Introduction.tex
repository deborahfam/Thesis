\chapter*{Introducción}\label{chapter:introduction}
\addcontentsline{toc}{chapter}{Introducción}

Desde su surgimiento en 2006, el aprendizaje profundo se ha establecido como una importante subdisciplina dentro del aprendizaje automático, especialmente en áreas relacionadas con la percepción visual humana. Esta metodología procesa datos a través de múltiples capas que incluyen estructuras complejas y transformaciones no lineales \brackcite{lecun2015deep}. Actualmente ha logrado avances significativos en áreas como la visión artificial, el reconocimiento de voz, el procesamiento del lenguaje natural, el reconocimiento de audio y la bioinformática \brackcite{deng2014deep}. Desde 2013, el aprendizaje profundo se ha reconocido como uno de los diez avances tecnológicos más significativos, dadas sus amplias aplicaciones potenciales en el análisis de datos \brackcite{cai2020review}.

El enfoque del aprendizaje profundo abstrae los datos en distintos niveles, lo que permite su aplicación en tareas complejas como la detección de objetos y la clasificación. Su capacidad para reemplazar la extracción manual de características por algoritmos eficientes de aprendizaje, ya sea no supervisada o semi-supervisada, ha revolucionado múltiples áreas \brackcite{song2013hierarchical}. Esta revolución incluye el campo de la atención médica, donde la gestión y análisis de un volumen abrumador de datos médicos es un desafío crucial.

En el ámbito de la atención médica, especialmente en la dermatología, el aprendizaje profundo ha mostrado un potencial extraordinario. La dermatología, que se enfoca en el estudio y tratamiento de enfermedades de la piel, se enfrenta al desafío del cáncer de piel, el tipo de cáncer más frecuente a nivel mundial. La detección precoz de este cáncer es vital \brackcite{american_cancer_society_estadisticas_2023}, y aquí es donde el aprendizaje profundo, con su habilidad para analizar y clasificar imágenes médicas con precisión, juega un rol transformador. La integración de estas tecnologías en la práctica dermatológica no solo mejora la precisión diagnóstica, sino que también promete revolucionar el tratamiento y manejo de diversas afecciones cutáneas \brackcite{fundacionpielsana_dermatologia}."

\section*{Motivación}

La motivación detrás del uso de algoritmos de \textit{machine learning} (ML) para el diagnóstico del cáncer de piel es significativa y valiosa. A diferencia de otros tipos de cáncer, el cáncer de piel se forma en la superficie de la piel y suele ser visible. Esto plantea una oportunidad única para la detección temprana y el tratamiento, lo cual es esencial ya que la mayoría de los casos de cáncer de piel son tratables si se detectan a tiempo \brackcite{american_cancer_society_estadisticas_2023}.

Estos algoritmos pueden identificar patrones complejos con una precisión y consistencia mayor que los métodos de diagnóstico humano, reduciendo así la posibilidad de diagnósticos incorrectos debido a la interpretación subjetiva y variable de los expertos \brackcite{das2021machine}. Además, el ML puede procesar grandes volúmenes de datos rápidamente y su capacidad para aprender y adaptarse con el tiempo significa que la detección y clasificación del cáncer de piel puede mejorar continuamente \brackcite{das2021machine}.

Aunque ya existen algoritmos eficientes para la clasificación de melanomas, una forma de cáncer de piel, el desarrollo de un modelo capaz de clasificar varios tipos de cáncer de piel y generalizar entre ellos es un objetivo crucial. Esto ampliaría el alcance de las imágenes dermatológicas procesables, mejorando potencialmente la precisión en la detección y el tratamiento de distintas formas de cáncer de piel."


\section*{Antecedentes}

El desarrollo de las Redes Neuronales Convolucionales (CNN) ha sido fundamental para la identificación de características en imágenes médicas, una base sobre la cual se construye la motivación actual para aplicar ML en la dermatología. Su uso en medicina ha demostrado ser eficaz para capturar patrones específicos en datos de imágenes con alta precisión (\brackcite{unal2023doc}). Por ejemplo, Brinker et al. \brackcite{brinker2018skin} en 2018, analizan 13 artículos sobre la aplicación de CNN en la clasificación de lesiones cutáneas, resaltando su alto rendimiento y la posibilidad de reutilizar CNNs pre-entrenadas. En 2020, Ameri et al. \brackcite{ameri2020deep} hacen un avance significativo al implementar una CNN profunda para procesar imágenes dermatoscópicas directamente, lo que mejora la eficacia del proceso de clasificación. Más recientemente, en 2022, Shetty et al. \brackcite{shetty2022skin} logran una precisión del 95,18\% en la clasificación de lesiones cutáneas utilizando CNN, demostrando su superioridad sobre otros algoritmos de ML.

En otras investigaciones enfocadas al diagnóstico mejorado de lesiones pigmentadas, Tajerian et al. (2023) \brackcite{tajerian2023design} presentan un enfoque metodológico para el diagnóstico de lesiones cutáneas pigmentadas utilizando CNN, logrando una alta puntuación F1 de 0,93. En 2021 Adegun et al. \brackcite{adegun2021deep} recoge un conjunto de estudios enfocados en el desarrollo de algoritmos con CNN para la detección de cáncer de piel. Estos incluyen enfoques de segmentación y clasificación, arquitecturas de auto-encoder-decoder, y la implementación de redes pre-entrenadas como AlexNet y VGG16.

Las universidades han desempeñado un papel crucial en la investigación y el desarrollo de tecnologías avanzadas en el campo de la medicina, especialmente en la detección y tratamiento del cáncer. Estos centros académicos no solo proporcionan una base sólida para la investigación teórica, sino que también fomentan la innovación práctica mediante el uso de tecnologías emergentes como el aprendizaje automático y la inteligencia artificial. En particular, nuestra universidad ha contribuido significativamente a este campo.

Las universidades han desempeñado un papel crucial en la investigación y el desarrollo de tecnologías avanzadas en el campo de la medicina, especialmente en la detección y tratamiento del cáncer. Estos centros académicos no solo proporcionan una base sólida para la investigación teórica, sino que también fomentan la innovación práctica mediante el uso de tecnologías emergentes como el aprendizaje automático y la inteligencia artificial. En particular, nuestra universidad ha contribuido significativamente a este campo.

Un claro ejemplo de esta contribución es la tesis de Darien Viera Barredo titulada \textit{Autómata celular estocástico en redes complejas para el estudio de la invasión, migración y metástasis del cancer} \brackcite{automata_celular_thesis} proporciona un marco detallado sobre cómo se aborda el estudio del cáncer desde una perspectiva matemática y computacional avanzada. El modelo propuesto utiliza autómatas celulares estocásticos para simular el crecimiento avascular y vascular del tumor. En el se aborda la complejidad del ciclo vital tumoral, destacando la importancia de su comprensión tanto para la investigación del cáncer como para la salud pública. Tradicionalmente, la modelación matemática y computacional se ha centrado en las etapas tempranas del desarrollo tumoral, donde la mortalidad es baja. Sin embargo, este estudio se enfoca en las fases avanzadas, que son críticas para la vida del paciente.

Complementando esta línea de investigación, la tesis reciente de Claudia Olavarrieta Martínez \brackcite{ensemble_thesis} propone un \textit{ensemble} de redes neuronales para clasificar imágenes dermatoscópicas en cuatro categorías: melanoma, carcinoma basocelular, carcinoma espinocelular y otros utilizando la técnica de transferencia de conocimientos en redes como VGG16, ResNet50 y EfficientNet B0.

\section*{Problemática}

Es entonces notorio que las investigaciones mencionadas utilizan técnicas de machine learning para llevar a cabo el proceso de clasificación. Esto tiene sentido dado que la problemática central en la detección del cáncer de piel radica en la necesidad de mejorar la precisión y rapidez del diagnóstico. Tradicionalmente, esta tarea recae en dermatólogos y el diagnóstico de melanoma, que depende de la evaluación clínica y los hallazgos clásicos en la biopsia de la lesión. Pero la inspección visual puede no ser suficiente para diferenciar lesiones benignas de tumores malignos, y aunque la biopsia de piel es el estándar de oro, es un procedimiento invasivo con limitaciones. Además, la experiencia, el costo y la disponibilidad son desafíos para el uso generalizado de herramientas no invasivas en el diagnóstico clínico \brackcite{das2021machine}.

\section*{Objetivos}

Este trabajo propone como objetivo fundamental el diseño y validación de un modelo predictivo basado en \textit{deep learning} para el diagnóstico del cáncer de piel mediante la clasificación de imágenes dermatológicas. El modelo diseñado y desarrollado se enfoca en clasificar imágenes, priorizando tanto la precisión de los resultados como la capacidad de generalización del modelo. Esto se concibe así con la idea de desarrollar trabajos posteriores que admitan otros datos de entrada.

Entre los objetivos específicos del proyecto se encuentran:

\begin{enumerate}
    \item Estudiar el estado del arte sobre las técnicas empleadas en el diagnóstico de imágenes dermatológicas y su efectividad.
    \item Crear un modelo de \textit{deep learning} que dado un conjunto de imágenes de cáncer de piel clasifique el tipo al que pertenecen.
    \item Decidir mediante un algoritmo de machine learning la mejor distribución de datos para entrenamiento del modelo e implementar mejoras potenciales al modelo e hiperparámetros para mejor precisión del mismo.
    \item Implementar técnicas de validación para evaluar la precisión del modelo.
\end{enumerate}

\section*{Contribuciones}

La metodología propuesta podría contribuir al desarrollo de un sistema de clasificación de imágenes de cáncer de piel más preciso, para luego ser utilizado en la práctica clínica en el diagnóstico de cáncer de piel. De esta forma podrían extenderse los algoritmos de clasificación existentes para el análisis de estos datos. 

\section*{Estructura de la tesis}

El contenido de la tesis se organiza de la siguiente forma. En el capítulo 1 se exponen las principales alternativas presentes en la literatura que se han desarrollado para la clasificación de imágenes. En el capítulo 2 presenta el modelo propuesto para la implementación de un sistema de clasificación de imágenes de cáncer de piel. En los capítulos 3 y 4 se describe  los algoritmos y técnicas utilizadas, se describen los experimentos realizados y se exponen los resultados obtenidos y se analiza la efectividad de estos. Finalmente, se presentan las conclusiones de la tesis y las recomendaciones para investigaciones futuras.