\chapter{Estado del Arte}\label{chapter:state-of-the-art}

El cáncer de piel es el más frecuente del mundo \textit{annadir cita} . A diferencia de muchos otros tipos de cáncer que se 
desarrollan internamente, el cáncer de piel se forma externamente y suele ser visible, lo que lo convierte en el "cáncer que se ve". Esta visibilidad subraya 
la importancia de los exámenes de la piel, tanto realizados por uno mismo como por dermatólogos.El diagnóstico y el tratamiento oportunos son primordiales. 
La mayoría de los casos de cáncer de piel son tratables si se detectan precozmente, lo que subraya la necesidad de identificar la enfermedad con prontitud. 
La detección precoz no sólo salva vidas, sino que también evita que la enfermedad se vuelva peligrosa, desfigurante o incluso mortal.

\section*{Teledermatología}

Las imágenes dermatológicas se utilizan para detectar el cáncer de piel, mediante el análisis de las lesiones cutáneas y la pigmentación. Esta es una técnica 
utilizada para examinar las lesiones cutáneas mediante un instrumento manual denominado dermatoscopio, que amplía la piel hasta 10 veces. Con el avance de la 
tecnología, la atención sanitaria ha experimentado una transformación. Se ha producido un aumento de dispositivos de diagnóstico y se ha desplazado hacia el 
desarrollo de nuevas competencias en estadística y psicología de la toma de decisiones médicas.

La teledermatología comenzó a ganar terreno en la década de 2000 como una manera de proporcionar consulta dermatológica a distancia. Un trabajo pionero 
es el de Whited et al \cite{whited2002teledermatology}, que exploró la eficacia de la teledermatología en 2002. El estudio constató que la teledermatología 
reducía significativamente el tiempo de intervención de los pacientes, con una mediana de 5 días para las derivaciones teledermatológicas frente a 28 días para 
las derivaciones tradicionales. El estudio también constató que la teledermatología era especialmente eficaz para pacientes con afecciones urgentes o semi-urgentes. 
El artículo destaca las posibles ventajas de la teledermatología para mejorar el acceso a la asistencia y reducir los tiempos de espera de los pacientes.
