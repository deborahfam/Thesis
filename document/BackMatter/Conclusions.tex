\begin{conclusions}
En el presente trabajo se dio solución al problema de clasificación de tumores de cáncer de piel mediante un algoritmo de machine learning. La metodología propuesta se basó en la utilización de una red neuronal convolucional pre-entrenada, EfficientNetB1, un algoritmo de ajuste de aprendizaje y varias capas de optimización adicionales. Para el entrenamiento y la validación del modelo, se utilizó el conjunto de datos HAM10000, que destaca por su diversidad y extenso alcance, abarcando una amplia gama de lesiones cutáneas pigmentadas.

Los resultados obtenidos son prometedores y sugieren un considerable potencial de los modelos de aprendizaje profundo para mejorar la precisión y eficiencia en el diagnóstico. El experimento 2, en particular, alcanzó una eficacia de clasificación del 87\%, que, aunque es inferior a métodos más robustos mencionados en el estado del arte, es prometedor, especialmente considerando que los algoritmos desarrollados utilizando EfficientNet generalmente alcanzan eficacias entre el 84\% y el 86.5\%.

Además, se destaca la importancia de un balance proporcional en el conjunto de datos utilizado (HAM10000). Se observó que la desproporción en este conjunto de datos lleva a errores de sobre-ajuste y a un rendimiento inferior del modelo. Esto indica que la calidad y la composición del conjunto de datos son aspectos críticos para el éxito de los modelos de aprendizaje profundo en aplicaciones de clasificación de imágenes dermatológicas.
\end{conclusions}
