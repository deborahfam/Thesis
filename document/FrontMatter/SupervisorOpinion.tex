\begin{opinion}

El trabajo “Aprendizaje automático orientado a la clasificación de cáncer de piel: un enfoque basado en EfficientNetB1” presentado por la estudiante Deborah Famadas Rodríguez constituye un resultado notable en el campo de la biomedicina, bioinformática. En este trabajo de diploma se presenta una metodología para la clasificación automática del cáncer de piel mediante el uso de técnicas de aprendizaje profundo. La solución propuesta se basa en el uso de los pesos obtenidos de una red neuronal convolucional pre-entrenada, EfficientNetB1. Este proceso se beneficia del aprendizaje de grandes conjuntos de datos y experiencias previas para mejorar el reconocimiento de patrones en imágenes dermatológicas ya adquirido por la red convolucional. Luego se hace uso de un algoritmo propio de learning rate para ajustar el aprendizaje del modelo. 

Para la implementación de este marco de trabajo, el diplomante debió asimilar un volumen considerable de información, de disimiles campos, entre ellos el referente al aprendizaje profundo establecido como una importante subdisciplina dentro del aprendizaje automático, especialmente en áreas relacionadas con la percepción visual humana. Esta metodología procesa datos a través de múltiples capas que incluyen estructuras complejas y transformaciones no lineales. Actualmente ha logrado avances significativos en áreas como la visión artificial, el reconocimiento de voz, el procesamiento del lenguaje natural, el reconocimiento de audio y la bioinformática. El aprendizaje profundo se ha reconocido como uno de los diez avances tecnológicos más significativos, dadas sus amplias aplicaciones potenciales en el análisis de datos. El enfoque del aprendizaje profundo abstrae los datos en distintos niveles, lo que permite su aplicación en tareas complejas como la detección de objetos y la clasificación. Su capacidad para reemplazar la extracción manual de características por algoritmos eficientes de aprendizaje ya sea no supervisada o semi-supervisada, ha revolucionado múltiples áreas. Esta revolución incluye el campo de la atención médica, donde la gestión y análisis de un volumen abrumador de datos médicos es un desafío crucial.

El trabajo escrito presenta una estructura clara y organizada que permite fácil comprensión de los contenidos incluidos. Además de presentar los resultados obtenidos en su trabajo, el diplomante presenta elementos técnicos acerca de aprendizaje automático, otros elementos referentes a la programación como un grupo de técnicas y algoritmos tanto de las ramas de inteligencia artificial como del campo de la optimización, los cuales utiliza en el desarrollo de su trabajo, lo cual hacen del documento escrito un buen material de referencia para los futuros trabajos relacionados con el tema.

La diplomante ha trabajado con gran dedicación durante toda su trayectoria, lo caracterizan su constancia y dedicación al trabajo de investigación. Es una excelente estudiante con gran pasión por la investigación. Debe destacarse que tuvo que asimilar en un período muy corto todo un número de conceptos, definiciones, referentes a temas de la tesis, manejar una bibliografía compleja. Además, fue capaz de manera convincente y resuelta de enfrentar las dificultades con independencia, de manera creativa.

Por todo lo antes expuesto, le propongo al tribunal la evaluación del presente trabajo de excelente y que le sea concedida a Deborah Famadas Rodríguez el título de licenciado en Ciencia de la Computación. \\

\textbf{Tutores:} Dr. Reinaldo Rodríguez Ramos,
Dr Yudivian Almeida Cruz.

\end{opinion}