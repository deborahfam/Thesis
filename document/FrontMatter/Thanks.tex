\begin{acknowledgements}
    De esta forma concluyen 5 años de duro y constante trabajo y aprendizaje que no hubieran sido posible sin cada una de esas personas que estuvieron siempre apoyándome en cada paso que daba, que nunca me dejaron rendirme y que si hoy estoy donde estoy, es gracias a ellos. Esta sección es para agradecerles.

    Bien saben mis padres que cuando pequeña no me despegaba un segundo de la computadora. Ese ímpetu por querer aprender cada vez más lo heredé de ellos. Por eso cuando tomé la decisión de coger una de las carreras más desafiantes en el mundo de la ciencia, no dudaron ni por un segundo en darme su apollo incondicional. No fue fácil la trayectoria, viniendo de una provincia no necesariamente céntrica, teniendo que estar becada en el 'Bahía', pasar noches sin dormir y alejada de la familia, cada día seguía recibiendo ese mensaje de mi mamá: 'Cuqui, cómo va todo?' que me sacaba del estrés momentáneo y volvía a experimentar la tranquilidad. A mi mamá, que siempre luchó por mí, incluso cuando ni me lo merecía, incluso ni cuando las fuerzas le daban, que me ha hecho la mujer que soy y la que seré, le regalo esto que hoy construí y mucho más.

    De las abuelas se dice que son las que consienten a los nietos. Yo fui la primera de cuatro, su ' negrita bonita' (ni me pregunten por qué porque ni sé) y la más consentida. A mi 'Aya' le dedico esto. Por siempre decirme ' Pero tú eres inteligente, eso tú lo apruebas' lo mismo si le hablaba de Análisis matemático de 2 variables a si le hablaba de ' Educación física'. Ella fue la primera que vio en mi, lo que nadie, siempre sonriente, una sonrisa con algunas ventanas, pero con dirección al alma.
    
    Esta tesis tiene un alo mágico, y es que un angelito me ha estado velando todo el tiempo. 'Eo', nombre que mi subconsciente de pequeña le asignó a mi abuelo, fue y será parte de mi siempre. Nunca seré capaz de olvidar su risa 'kkkk' mientras me jalaba las orejas y me decía 'Ere una caballaaaa'. Él también se merece todo.

    A mi padre, que sabe amar a su forma única. El primero que me sentó delante de una computadora, que me cargaba a caballito y que al día de hoy sigue luchando por que yo salga adelante. A mis tíos, tías, primas, etc, con especial mención a mi tío 'Rene' que fue como un segundo padre para mí, y siempre estuvo cuando lo necesitábamos.

    Y continuando el tema de los agradecimientos familiares. Sería incorrecto de mí decir 'Le quiero agradecer a mis amigos', cuando son familia practicamente.

    Alguna vez se han fijado que somos muy poco agradecidos con el dedo pequeño del pie. Ese dedo que te ayuda a identificar obstáculos, incluso si tus ojos no pueden verlos (sobre todo si tus ojos no pueden verlos). Que te da estabilidad y soporte, que es gordito y aunque tiene 4 dedos más a los lados sigue siendo diferente. Yo tengo una persona así en mi vida, el 'Pocho' o como la gente lo llama 'Jean Pierre'. Es inconcluso de mi parte dedicarte un texto que no se empape en las lágrimas que derramaré si lo escribo, dado que solo tú y yo sabemos que significas. Te agradezco por tantas noches, tantas mañanas y tantas tardes, que no solo fuiste y eres amigo, mentor, el contacto más contactado y más importante de mi whatsapp, fuiste y eres TODO. Este mérito es completamente tuyo, esto y la libreta de discreta que tengo en la casa que a ver cuándo te la devuelvo jjj.
    
    A mi otra familia, Omar y Andy, que me apadrinaron desde primer año, que me han visto crecer y yo a ellos, que me dieron el primer cocotazo de la carrera y que hacen muchas veces mejor mis días. A David, porque no me lo perdonaría nunca si no te dedico unas líneas, contigo también fueron noches, incansables e incesantes de puro estrés al recibir por cuarta vez de mi la frase 'Ay que no te entiendo' y tú, tan calmado y confiado como siempre volver a la carga, siempre creyéndome capaz, sino como un hombre tan ocupado como tú hubiera dedicado tanto a una causa perdida jjj. A tí que fue muy importante ese abrazo luego de esa prueba de Eda, para tí también va esto.

    Comenta la gente que uno no se debe enamorar de sus mejores amigos, que rompe la amistad y que luego terminas arrepintiéndote. Pero una no puede cohibirse ante lo inevitable. Si bien besaste tú primero, no podría estar más agradecida al respecto. Y es que no hay otra voz más dulce que pueda calmar mi tormenta. Ay gaby, que más podría decirte que no haya dicho ya, que siempre estuviste ahí, a pesar que casa de lía te daba alergia, a pesar que solo tenías una muda de ropa, a pesar de todo, ibas, y solo el hecho de tenerte ahí era suficiente, es suficiente. Esto tampoco hubiera sido posible sin tí, y aunque lo dudes intensamente, te lo digo de corazón.

    Y quiero agradecer a todos, a mis amigos del pre, por decirme 'Si mija coge eso, si al final, loca ya estás', a los amigos de la carrera, a Lachi, Alejandra, Roxana, Ana y Toni por sacarme de mi algarabía y disfrutar conmigo en las fiestas, por reírse de mis boberías en las conferencias y por siempre estar igual de estresados que yo con los exámenes. A mis suegros, que tan solo 10 meses en mi vida, mira que si han causado hueco sentimental. Y a todos aquellos que por falta de espacio (No me puedo pasar de 100 páginas en la tesis) o de memoria me ayudaron a lograr esto que hoy consigo.


\end{acknowledgements}