\begin{resumen}		
	En este trabajo de diploma se presenta una metodología para la clasificación automática del cáncer de piel mediante el uso de técnicas de aprendizaje profundo. La solución propuesta se basa en el uso de los pesos obtenidos de una red neuronal convolucional pre-entrenada, EfficientNetB1. Este proceso se beneficia del aprendizaje de grandes conjuntos de datos y experiencias previas para mejorar el reconocimiento de patrones en imágenes dermatológicas ya adquirido por la red convolucional. Luego se hace uso de un algoritmo propio de \textit{learning rate} para ajustar el aprendizaje del modelo. Para reforzar la eficiencia, se incorporan capas de normalización, densas, de regularización y \textit{dropout}, además de una capa de salida. Además, se realizan una serie de experimentos donde se evalúa el impacto de la division y normalización de datos con respecto a la eficiencia de clasificación del algoritmo.

\end{resumen}

\begin{abstract}	
	In this diploma work, a methodology for automatic skin cancer classification using deep learning techniques is presented. The proposed solution is based on the use of weights obtained from a pre-trained convolutional neural network, EfficientNetB1. This process benefits from the learning of large datasets and previous experiences to improve the pattern recognition in dermatological images already acquired by the convolutional network. A proprietary learning rate algorithm is then used to adjust the model learning.	To reinforce efficiency, normalization, dense, regularization and dropout layers are incorporated, as well as an output layer. In conjunction with this, a series of experiments are carried out, where the impact of the division and normalization with respect to the classification efficiency of the algorithm is evaluated.
\end{abstract}