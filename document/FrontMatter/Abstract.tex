\begin{resumen}
	El Machine Learning (ML) o Aprendizaje Automático (AM) es un subcampo de la Inteligencia Artificial (IA),
que se define como la capacidad de una máquina para aprender, mejorando su rendimiento. Los avances en
las tecnologías de IA y ML prometen. El potencial de estas herramientas para generar conocimiento a partir de cantidades masivas
de datos es enorme, pudiendo así ayudar a tomar decisiones, que incluirán intervenciones, y tratamientos
de precisión contra el cáncer. La detección de cáncer de piel utilizando Redes Neuronales Convolucionales (CNN) 
es un área prometedora en el campo de la inteligencia artificial y el aprendizaje automático. Las CNN son especialmente
adecuadas para el procesamiento de imágenes y han demostrado un alto rendimiento en tareas de clasificación y detección
de objetos. Este trabajo de diploma presenta una serie de implementaciones, las que incluyen CNN, el modelo EfficientNetB1 
y el uso de Learning Rate Adjustment a un Dataset de imágenes para lograr una detección efectiva de diferentes tipos de cáncer 
de piel a partir del entrenamiento de una IA. Estos resultados pueden sentar las bases para futuras investigaciones y mejoras 
en el campo de la detección temprana del cáncer de piel utilizando técnicas de aprendizaje automático y 
redes neuronales convolucionales.
\end{resumen}

\begin{abstract}
	Machine Learning (ML) or Machine Learning (ML) is a subfield of Artificial Intelligence (AI), which is defined 
as the ability of a machine to learn, improving its performance. Advances in AI and ML technologies hold promise. 
The potential of these tools to generate knowledge from massive amounts of data is enormous, thus being able to help 
make decisions, which will include interventions, and precision in cancer treatments. The detection of skin cancer using 
Convolutional Neural Networks (CNN)  is a promising area in the field of artificial intelligence and machine learning. 
CNNs are particularly well suited for image suitable for image processing and have demonstrated high performance in object 
classification and object detection tasks. This diploma work presents a number of implementations, which 
include CNNs, the EfficientNetB1 model, and the use of Learning Rate Adjustment to an image dataset to achieve 
effective detection of different types of skin cancer by training an from the training of an AI. These results can lay 
the foundation for future research and improvements in the field of early skin cancer detection using machine learning 
techniques.
\end{abstract}