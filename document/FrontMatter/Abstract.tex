\begin{resumen}	
	El Machine Learning (Aprendizaje Automático) es una disciplina del campo de la Inteligencia Artificial que, a través de algoritmos, dota a los ordenadores de la capacidad de identificar patrones en datos masivos y elaborar predicciones \brackcite{Iberdrola2023MachineLearning}. Las Redes Neuronales Convolucionales (CNN), una especialización de esta disciplina, conocidas por su eficacia en el procesamiento de imágenes, son ideales para detectar características comunes en imágenes dermatológicas. 
	
	En este trabajo de diploma se presenta una metodología para la clasificación automática del cáncer de piel mediante el uso de técnicas de aprendizaje profundo. La solución propuesta se basa en el uso de los pesos obtenidos de una red neuronal convolucional pre-entrenada, EfficientNetB1. Este proceso se beneficia del aprendizaje de grandes conjuntos de datos y experiencias previas para mejorar el reconocimiento de patrones en imágenes dermatológicas ya adquirido por la red convolucional. Luego se hace uso de un algoritmo propio de learning rate para ajustar el aprendizaje del modelo.

	Para reforzar la eficiencia, se incorporan capas de normalización, densas, de regularización y dropout, además de una capa de salida. Conjunto a esto se realizan una serie de experimentos de los que se exponen en este trabajo 2 donde se evalúa el impacto de la division y normalización de datos con respecto a la eficiencia de clasificación del algoritmo.

\end{resumen}

\begin{abstract}
	Machine Learning is a discipline in the field of Artificial Intelligence that, through algorithms, gives computers the ability to identify patterns in massive data and make predictions. Convolutional Neural Networks (CNN), a specialization of this discipline, known for their efficiency in image processing, are ideal for detecting common features in dermatological images. 
	
	In this diploma work, a methodology for automatic skin cancer classification using deep learning techniques is presented. The proposed solution is based on the use of weights obtained from a pre-trained convolutional neural network, EfficientNetB1. This process benefits from the learning of large datasets and previous experiences to improve the pattern recognition in dermatological images already acquired by the convolutional network. A proprietary learning rate algorithm is then used to adjust the model learning.

	To reinforce efficiency, normalization, dense, regularization and dropout layers are incorporated, as well as an output layer. In conjunction with this, a series of experiments are carried out, which are presented in this paper 2, where the impact of the division and normalization with respect to the classification efficiency of the algorithm is evaluated.
\end{abstract}